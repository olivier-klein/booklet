Pure spin currents -- i. e., directed flows of spin angular momentum -- are a fascinating manifestation of spin physics in the solid state. Pure spin currents can propagate not only in metals and semiconductors, but also in magnetically ordered insulators. This makes a whole new set of materials and material combinations interesting for spin transport experiments and spin-electronic devices. 
A prototypical example for this development is the spin Hall magnetoresistance (SMR) effect, which arises in magnetic insulator/normal metal hybrid structures owing to the flow of spin angular momentum across the magnet/metal interface. In the last few years, the SMR response of ferrimagnetic insulators with both collinear as well as non-collinear magnetic structure has been extensively investigated. In addition, SMR-like experiments in non-local geometry in magnetic insulator/normal metal nanostructures enable the electrical quantification of the magnon diffusion length. The insights obtained from these experiments pave the way for studying pure spin current transport also in antiferromagnets. In the talk, I will briefly review the SMR response characteristic of collinear and non-collinear ferrimagnetic phases, and then discuss our recent magneto-transport experiments in antiferromagnetic insulator/Platinum heterostructures. In particular, I will critically compare our experimental data with model calculations, and address the evolution of the magneto-resistance across the Neel temperature of the antiferromagnet.