In the past years, it was realized that the experimental methods allowing for the detection of single spins in the solid-state, which were initially developed for quantum information science, open new avenues for high sensitivity magnetometry at the nanoscale. In that spirit, it was proposed to use the electronic spin of a single nitrogen-vacancy (NV) defect in diamond as an atomic-sized magnetic field sensor. This approach promises significant advances in magnetic imaging since it provides non-invasive, quantitative and vectorial magnetic field measurements, with an unprecedented combination of spatial resolution and magnetic sensitivity under ambient conditions.

In this talk, I will illustrate how scanning-NV magnetometry can be used as a powerful tool for exploring condensed-matter physics, focusing on chiral spin textures in ultrathin magnetic materials.