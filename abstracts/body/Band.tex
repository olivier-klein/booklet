Recently, it was predicted that the sensitivity of a precessing magnetic needle magnetometer can surpass that of present state-of-the-art magnetometers by several orders of magnitude \cite{Jackson_Kimball_2016}.  In order to model the dynamics of a single-domain magnetic needle in a magnetic field, the dissipation of spin components not along the axis of easiest magnetization must be taken into account \cite{Gilbert_2004}. Where there is dissipation, fluctuations are also present \cite{Callen_1951}.  Fluctuations are a source of uncertainty that can affect the accuracy of the magnetometer.  We calculate the dynamics of a magnetic needle in the presence of an external magnetic field and determine the uncertainty of a magnetic needle magnetometer due to the fluctuations that give rise to Gilbert damping \cite{Gilbert_2004}, i.e., interactions with internal degrees of freedom such as lattice vibrations, spin waves, and thermal electric currents.  We solve the Heisenberg equations of motion for the spin, ${\hat {\bf S}}$, the unit vector in the direction of the axis of easiest magnetization, ${\hat {\bf n}}$, and the total angular momentum,  ${\hat {\bf J}}$, in mean-field approximation \cite{Band_2013}  by taking quantum expectation values of the equations, and noting that for large spin and orbital angular momentum, the standard deviation is small compared to the quantum average.   When fluctuations are included, numerical solution of the stochastic equations is difficult when the anisotropy coefficient is large.  Therefore we develop a perturbative solution around the adiabatic solution.  Analysis of the uncertainty in such magnetic field measurements in the limit of small and large magnetic fields will be presented.