Magnetic resonance force microscopy (MRFM) is a technique that reconstructs the 3D density distribution of nuclear spin species in nanoscale samples, such as individual macromolecules. While the feasibility of the method has been demonstrated \cite{Degen_2009}, state-of-the-art experiments have not yet reached the subnanometer regime required to reveal detailed molecular structures.

MRFM experiments operate at the physical boundaries of technology where every improvement in resolution must be earned by overcoming a manifold of physical and technical problems. Of crucial importance are, for instance, a stable feedback control of the mechanical resonator, a superb displacement detection sensitivity, suppression of standing waves in the rf-circuit and clever NMR pulse shaping for evading unwanted electrostatic interaction, and robust spin inversion protocols. Our poster summarizes our progress in MRFM technology and demonstrates first signal scans of single isotope-labeled influenza virus particles.