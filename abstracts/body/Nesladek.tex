Scalable principles for quantum state readout are one of open questions in quantum technology. Building on the recent results of photoelectric detection of magnetic resonances (PDMR) \cite{Bourgeois_2015,Gulka_2017} we discuss the prospect of PDMR technique for solid-state qubit devices in diamond. One of the key PDMR advantages over the optical detected magnetic resonance  (ODMR) technique are the high detection rates $\sim 5 \times 10^9$ sec$^{-1}$, exceeding orders of magnitude ODMR. Consequently, the novel detection technique might lead to the single shot readout of the NV centres quantum state and provide fast data acquisition at room temperature.  To achieve this goal we discuss the photoelectric gain, used to obtain high detection rates on small NV spin qubit ensembles. The photophysics of the transitions on NV centre is reviewed and several scenarios for obtaining highest single/noise ratio are presented. We demonstrate pulsed PDMR measurements, compatible with coherent manipulation of the spin states realized on quantum chips. Finally we realize a single NV qubits read electrically.
We discuss another defects such as NiV as a potential candidate for electrically read spin qubits in diamond.