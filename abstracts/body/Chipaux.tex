Nitrogen-Vacancy (NV) centers in diamond are promising probes to perform the detection of nuclear or electronic spin resonance with a nanoscale resolution. In a clean environment with high quality diamonds their unprecedented sensitivity already enabled the detection of single electron spins or of a few nuclear spins in a nanoscale voxel. In order to perform such measurement in vivo at the subcellular level, it has been proposed to use Fluoresent NanoDiamonds (FNDs) with NV centers directly inside living cells.
A first step is to control the uptake of these diamond nano-sensors by the targeted cells. While macrophages can spontaneously ingest them, most cells do not and may even have a cell wall for preventing it. In this talk I will present different methods for diamond uptake depending on the cell type \cite{Hemelaar_2017,Zheng_2017}. We validate the protocols with optical and electron microscopy that reveals where exactly the FNDs end up and what their fate is during and after the uptake, evidencing how these FNDs can be used as multipurpose labels \cite{Hemelaar_2017a} in living cells. Finally, we will also measure how and if these uptake methods and the magnetic resonance detection protocols influences the cell biology and we will propose methods for minimizing the effect.