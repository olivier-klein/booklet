The nitrogen vacancy (NV) center in diamond is an atomic-scale defect in diamond that is highly sensitive to a wide variety of fields: magnetic, electric, thermal, and strain. Here I discuss an NV-based imaging platform (Fig. 1) where we have incorporated an NV center into a scanning probe microscope and used it to image vortices in superconductors and skyrmions, nanoscale topological spin textures, in thin film magnetic multilayers. I also discuss recent experiments that utlize the NV center's sensitivity to fluctuating magnetic fields to image the conductivity of a proximal sample with nanoscale spatial resolution. A grand challenge to improving the spatial resolution and magnetic sensitivity of the NV is mitigating surface-induced quantum decoherence, which I will discuss in the second part of this talk. Decoherence at interfaces is a universal problem that affects many quantum technologies, but the microscopic origins are as yet unclear. With its sensitivity to electric and magnetic fields over a wide range of frequencies, we have used the NV center as a noise spectrometer to spectroscopically probe sources of surface-related decoherence, differentiating between electric and magnetic origins. These studies guide the ongoing development of quantum control and diamond surface preparation techniques, pushing towards the ultimate goal of NV-based single nuclear spin imaging.