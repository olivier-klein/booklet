Level repulsion (the opening of a gap between two degenerate modes due to coupling)
  is ubiquitous anywhere
  from solid state theory to quantum chemistry. 
  In contrast, 
  the mode frequencies can attract instead
  if, for instance,
  one of the modes has negative energy.
  The frequencies converge and develop imaginary components,
  leading to an instability;
  an exceptional point marks the transition.
  This, however, only occurs if
  the dissipation rates of the two modes are equal.
  We expose a theoretical framework for the process and
  realize it experimentally
  through engineered dissipation
  in a multimode superconducting microwave optomechanical circuit~%
  \cite{1709.02220v1}.
  Level attraction is observed 
  for a mechanical oscillator and a superconducting microwave cavity.
  An auxiliary cavity is used to tune the effective
  mechanical dissipation rate to match that of the the microwave cavity
  using sideband cooling.
  Two exceptional points are demonstrated that
  could be exploited for their topological properties.