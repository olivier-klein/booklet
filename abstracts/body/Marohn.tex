I will describe my team's progress exploiting magnet-tipped attonewton-sensitivity cantilevers \cite{Longenecker_2011,Longenecker_2012} to image biological assemblies and as-fabricated devices.   We have integrated magnet-tipped cantilevers into a scanned-probe microscope having three-dimensional coarse and fine positioning, operating at 4.2 kelvin, and having a 0 to 9 tesla magnetic field.  We have carried out magnetic resonance experiments on a 200 nm thick film of polystyrene doped with a nitroxide free radical.   This sample was spun-cast 
onto a coplanar waveguide (10 micrometer centerline, operating from dc to 40 Ghz).   We could significantly reduce surface-related cantilever frequency noise by coating the sample with a 10 nm metal layer and applying a voltage to the cantilever.  To align the cantilever with the coplanar waveguide's centerline we employ ``edge-finder'' cantilever frequency shifts fortuitously present when a voltage bias is applied to the waveguide's centerline \cite{1710.01442v1}.  We used force-gradient detection (e.g., frequency shifts) to sensitively detect Curie-law proton magnetization at 6 T, electron-spin resonance at 0.6 T, and hyperpolarized proton magnetization created by cross-effect dynamic nuclear polarization (DNP) at 0.6 T \cite{Issac_2016,Isaac_2017}.   To assess the potential improvements in MRFM imaging obtainable with DNP, we have developed a theory of the signal-to-noise ratio covering MRFM experiments detecting magnetization fluctuations (having a random sign) and Curie-law and hyperpolarized magnetization (with a well-defined sign).   We have developed rapid Bayesian protocols for reconstructing proton-density images of isolated proteins from force-fluctuation maps and for determining the location of individual electron spins from frequency-shift maps \cite{Moore_2009}.