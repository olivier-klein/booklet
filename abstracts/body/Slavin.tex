We propose  a generator of THz-frequency signals based on a layered structure consisting of a current-driven platinum (Pt) layer and a layer of an antiferromagnet (AFM) with easy-plane anisotropy where the magnetization vectors of the AFM sublattices are canted inside the easy plane
due to  the Dzyaloshinskii-Moriya interaction (DMI) \cite{1707.07491v1}. The DC electric current flowing in the Pt layer creates, due to the spin-Hall effect, a perpendicular spin current which, being injected into the AFM layer, tilts the DMI-canted AFM sublattices out of the easy plane, thus exposing them to the action
of a strong internal exchange magnetic field of the AFM. The sublattice magnetizations along with the small net magnetization vector of a canted AFM start to rotate about the hard  anisotropy axis of the AFM with the THz frequency proportional to the injected spin current and a square root of the AFM exchange field. The rotation of the small net magnetization results in the THz -frequency dipolar radiation that can be directly received by an adjacent (e.g. dielectric) resonator. We demonstrate theoretically that the generated frequencies in the range $f = $0.05- 2.0 THz are possible at the experimentally reachable magnitudes of the driving current density, and evaluate the
power of the signal radiated into different types of resonators, showing that this power increases with the increase of the generation frequency $f$.