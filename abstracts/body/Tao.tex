Magnetic nanostructures have played a central role in the development of magnetic resonance force microscopy (MRFM) \cite{Mamin_2007,Poggio_2007,Nichol_2012} and force-based detection of magnetic moments \cite{Tao_2016}.  Access to highly magnetic tips with precise, nanometer-resolution controls of tip geometry and material composition would further advance the technique by enabling closer approach of the gradient source to the sample \cite{Kuehn_2006,Overweg_2015}.
 
One approach to controllable tip geometries is the bottom-up growth of single-crystalline magnetic nanowires \cite{Chan_2010}.  However, the production of nanowire materials, uniformly oriented along any arbitrarily chosen crystal orientation, is an important, yet unsolved, problem in material science \cite{Dasgupta_2014}. A practical need for MRFM has thus led us to devise a generalizable solution to this material science problem, using FeCo as the demonstration material system. 

We found a solution is based on the technique of glancing angle deposition \cite{Hawkeye_2007} combined with a rapid switching of the deposition direction between crystal symmetry positions.  We showcase the power and simplicity of the process in one-step fabrications of $<1 0 0>$, $<1 1 0>$, $<1 1 1>$, $<2 1 0>$, $<3 1 0>$, $<3 2 0>$ and $<3 2 1>$-oriented nanowires, three-dimensional nanowire spirals, core-shell heterostructures and axial hybrids.  The resulting nanowires are single-crystals, have high saturation magnetization of 2.0(2) Tesla, and passivated by a surface oxide below 3 nm in thickness after one year of storage in air.  Our results provide a new capability for tailoring the shape and properties of nanowires, should be generalizable to any material that can be grown as a single-crystal biaxial film, and has already offered a new route towards next-generation tip-on-cantilever MRFM sensors.