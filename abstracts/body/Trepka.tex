Powerful and controllable magnets are central to magnetic resonance imaging (MRI). Magnetic resonance force microscopy (MRFM) is a version of nanoscale MRI capable of non-invasive three-dimensional atomic imaging of single samples \cite{Degen_2009,Nichol_2013}.  MRFM signal strength scales quadratically with magnetic field gradient, and imaging time scales to its 4th power \cite{Degen_2009}.  Access to highly magnetic tips are thus crucial for the advancement of MRFM toward and beyond single-proton sensitivity.

Current state of the art magnetic tips are made of amorphous or polycrystalline dysprosium and iron-cobalt alloys \cite{Mamin_2012,Longenecker_2012}. While iron-based alloys are reaching their limit imposed by an intrinsically lower magnetic moment of 2.2 $\mu_B$ \cite{Longenecker_2012,Shamsudhin_2016}, technology, namely the quality of nanoscale material quality that is achievable by e-beam evaporation onto room-temperature substrates, has limited rare-earth metals.    This is unfortunate as the rare-earth holmium (Ho) has the highest magnetic moment of any naturally occurring element, $10.6 \mu_B$, and a bulk volume magnetization nearly twice as high as that of iron nanowires \cite{Rhodes_1958}.  To push the nanotechnology for ultra-high-moment nanomagnets toward nature's limit, we develop synthetic routes toward metallic holmium and dysprosium nanowires.  We select vacuum vapor deposition at elevated temperatures as a promising route towards these target nanostructures\cite{Horprathum_2014,Eames_2006,Chen_2007}.  Work toward synthesizing ferromagnetic holmium nanowires via substrate-directed processes will be presented.