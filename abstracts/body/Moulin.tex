The goal of my PhD is to develop a quantitative and ultra-sensitive magnetic microscope which combine a magnetoresistive sensor and a canning probe microscope. One of the main interest of the microscope is that, by applying a small AC field, the magnetic susceptibility of surface could be measured and used for non-destructive testing of steel. Indeed, studies have shown that the magnetic susceptibility of steel is correlated to their mechanical properties \cite{M_sz_ros_2006}.
The microscope is composed by a giant magnetoresistive (GMR) sensor  integrated at the apex of the flexible cantilever. The process of micron-size MR-sensors has been developed in the lab, but in order to improve spatial resolution, I am currently developing a process to make nanometer size GMR sensors.  The fabricated sensors will then be characterized  in term of noise and sensitivity. One of the issues is to conserve the sensitivity and noise level while decreasing the size of the sensor. We are investigating different GMR stack structures and especially working on the free layer.