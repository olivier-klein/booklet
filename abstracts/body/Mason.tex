Modern optomechanics has been enabled by the availability of high quality mechanical resonators.  Here, we present ongoing work with recently-pioneered \cite{Tsaturyan_2017} 'ultracoherent' membranes, which exploit phononic engineering and soft-clamping techniques to achieve unprecedented mechanical coherence.  With Q-factors of $10^8$ at room temperature and nearly $10^9$ at T$\sim$10K, these devices enable novel experiments in various regimes.

First, by moderate cryogenic cooling, we can readily achieve quantum cooperativities $C_Q = 4g^2/\kappa\gamma \gg 1$, enabling quantum protocols with a mechanical system whose thermal decoherence rate ($\gamma$) approaches that of trapped ions ($\sim$1 ms).  This should allow, for instance, stroboscopic QND localization of a single mechanical mode, or using a novel multimode device, two-mode mechanical squeezing and entanglement.  Remarkably, these membranes can even achieve $C_Q>1$ without cryogenic cooling, and by integration with low-noise fiber mirrors, achieve a macroscopic room-temperature system capable of quantum behavior.

We also exploit the high Q of these resonators to advance the capabilities of various hybrid systems.  These include electro-optomechanical systems\cite{Bagci_2014} as well as systems in which the membrane motion couples to the collective spin of a room-temperature atomic ensemble\cite{M_ller_2017}.  Finally, our soft-clamping techniques can also be exploited to advance force sensing applications.  Towards this end, we have developed low-mass soft-clamped resonators, which seek to advance scanning force microscopy (in particular, MRFM) by moving beyond the usual tip-as-sensor paradigm, and exploiting our ultracoherent mechanical devices.