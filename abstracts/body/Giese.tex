Over the last decade the unique properties of the negatively charged nitrogen vacancy color center (NV-) in single crystalline diamond have been explored for developing new types of scanning probe and wide-field magnetometers with high sensitivity and spatial resolution. The realization of magnetometric tips is challenging and will be the topic of this presentation.
Homoepitaxial growth of ultra pure diamond on single crystalline diamond substrates has been extensively developed at IAF\cite{Widmann_2016}. The microwave plasma assisted CVD process is carried out in an UHV load-lock system where methane is purified by means of a zirconium gas-purification system in combination with palladium hydrogen purification. After homoepitaxy the substrates are lasercut and scaife-polished to yield 30-50$\mu$m membranes. Diamond vertical waveguide tips are fabricated via a top-down process by use of highly anisotropic plasma dry etching2. These tips typically exhibit diameters of 200 nm with lengths ranging from 2 to 4 $\mu$m\cite{Widmann_2015}. Shallow NV centers at the apex were realized by Nitrogen implantation or by growth of Nitrogen delta-doped diamond layers in combination with thermal annealing. Nitrogen concentration can be controlled from the lower ppb level to 500ppm for delta layers of typically 20nm thickness, yielding high $T_2$ times above 100$\mu$s. Several hundred micro-optical tips are realized per membrane. Tips are singulated and transferred to quartz tuning fork AFM probes by use of micro manipulators (Imina Technolgies SA). For characterization we use scanning electron microscopy (FIB-SEM) and confocal micro-photoluminescence spectroscopy ($\mu$-PL). We will discuss and summarize our results with respect to scalable tip fabrication technology as well as to novel approaches for higher spin coherence times.