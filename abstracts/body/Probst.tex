Electron spin resonance (ESR) spectroscopy is widely employed for the detection and characterization of paramagnetic species and their magnetic and chemical environment. In a classical ESR spectrometer, the spins precess in an external magnetic field and emit small microwave signals into a cavity, which are amplified and measured. In this work \cite{Probst_2017}, we make use of the toolbox of circuit quantum electrodynamics to boost the sensitivity of such a spectrometer by many orders of magnitude to the level of 65 spins/$\sqrt{\text{Hz}}$ with a signal-to-noise ratio of 1.
This is achieved by using a low impedance, high quality factor superconducting micro-resonator in conjunction with a Josephson parametric amplifier operated below 20 mk \cite{Bienfait_2015}. The energy relaxation time T$_1$ of the spins (Bi donors in $^{28}$Si) is limited by the Purcell effect to 20 ms allowing fast repetitive measurements while the coherence time T$_2$ is approximately 1.7 ms. 
The necessarily narrow bandwidth of our resonator distorts the shape of our short drive pulses and in order to control the spin ensemble more accurately, we employ shaped compensation pulses canceling the filtering effect of the cavity. Then, we perform electron spin echo envelope modulation (ESEEM) experiments at low magnetic fields allowing us to detect weakly coupled $^{29}$Si nuclear spin impurities in our $^{28}$Si sample.
The sub pico-liter detection volume of our spectrometer makes it an interesting tool for investigating paramagnetic surfaces and, in particular, recently discovered 2D materials.