While ferromagnets are at the heart of daily life applications, the large magnetization and energy costs for switching bring into question their suitability for reliable low-power spintronic devices. Non-colinear antiferromagnetic systems do not suffer from this problem and often possess remarkable extra functionalities: non-collinear spin order\cite{Coey_1987}  may break space-inversion symmetry\cite{Kimura_2003,Cheong_2007} and thus allow electric field control of magnetism\cite{Lottermoser_2004,Heron_2014}, or produce emergent spin-orbit effects\cite{Nayak_2016,Hanke_2016} which enable spin-charge interconversion\cite{Zhang_2016}. To harness these unique traits for next-generation spintronics, the equivalent nanoscale control and imaging capabilities that are now routine for ferromagnets must be developed for antiferromagnetic systems. In this work, using a non-invasive scanning probe based on a single nitrogen-vacancy (NV) defect in diamond\cite{Maze_2008}, we demonstrate the first real-space visualization of non-collinear antiferromagnetic order in a magnetic thin film, at room temperature. We image the spin cycloid of a multiferroic BiFeO3 thin film and extract a period of $\approx$~70nm, consistent with values determined by macroscopic diffraction, in addition we take advantage of the magnetoelectric coupling present in BiFeO$_3$ to manipulate the cycloid propagation direction by an electric field. These results demonstrate how BiFeO$_3$ can be used as a versatile platform for the design of reconfigurable nanoscale spin textures. This also proves that this technique can be further used in investigating exotic materials and magnetic orders.