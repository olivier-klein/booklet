The technique of cavity optomechanics enables enhanced readout of mechanical motion \cite{Aspelmeyer_2014}.  When applied to low-moment of inertia torsional resonators, cavity optomechanics results in improved sensitivity for torque magnetometry.  We have recently demonstrated the integration of a cavity optomechanical resonator with a nanoscale ferromagnetic structure.  Using this system, we have been able to measure calibrated mechanical torques as small as 32 zNm, corresponding to sensing an external field of 0.12 A/m (150 nT).  Beyond simply sensing magnetic fields, we have been able to quantitatively determine the magnetic properties of the integrated iron nanostructure, such as the magnetization at the room temperature - which was not known a priori and yet matches mumax simulations of the magnetization when polycrystalline disorder is included.

Going forward, we are interested in using the sensitivity to explore torque mixing spectroscopy in nanomagnetic materials with lower magnetic susceptibility than permalloy \cite{Losby_2015}.  Preliminary measurements show that we can observe GHz spin dynamics in iron structures, when biased near the point of magnetic switching, which are consistent with magnetic simulations.  We will also discuss the experiments on using this triply resonant system of optical, mechanical, and magnetic resonators, for the purpose of phase sensitive microwave to telecom wavelength conversion.