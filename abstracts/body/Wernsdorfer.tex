One hundred years ago it has been discovered that a change of magnetization in a macroscopic magnetic object results in a mechanical rotation of this magnet. The effect, known as Einstein de Haas or Richardson effect, demonstrates that a spin angular momentum in the magnet compensates for the mechanical angular momentum associated with its rotation. The experiment is therefore a macroscopic manifestation of the conservation of total angular momentum and energy in eletronic spins. According to Noether's theorem, conservation of angular momentum follows from a system`s rotational invariance and would be valid for the ensemble of spins in a macroscopic ferromagnet as well as for an individual spin. It has been recently proposed that single spin systems would therefore manifest an Einstein de Haas effect at the quantum level.\cite{Chudnovsky_1994,Chudnovsky_2005,Garanin_2011} 
Here we propose the first experimental realization of a quantum Einstein-de Haas experiment and describe a macroscopic manifestation of the conservation of total angular momentum in individual spins, using a single molecule magnet coupled to a nanomechanical resonator. We demonstrate that the spin associated with the single molecule magnet is then subject to conservation of total angular momentum and energy which results in a total suppression of the molecule's quantum tunneling of magnetization.\cite{Ganzhorn_2016}