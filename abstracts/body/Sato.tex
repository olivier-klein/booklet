Optical vortex (or vortex beam) is the singular laser beam carrying an intrinsic orbital angular momentum, \cite{Allen_1992} and has been actively studied especially as a new tool to control various kinds of matter. Due to the presence of the angular momentum, the vortex beam has a characteristic spatial distribution of electromagnetic fields. Making use of the spatial features, we theoretically propose new ways of generating/controlling magnetic nanostructures such as skyrmions and skyrmioniums \cite{Fert_2013} in magnets with vortex beams. We concentrate on two set ups: Applications of high- (ultraviolet, visible) and low-frequency (Tera Hz: THz) vortex beams to magnets. In the case of the high-frequency beam, its oscillation is too faster for typical spin dynamics and the spins cannot be directly coupled to the oscillating fields, while spins can feel heating driven by the beam. Based on Landau-Lifshitz-Gilbert (LLG) equation with the beam-driven heating, we show that vortex beams in ultraviolet or visible range can generate multiple ring-shape magnetic defects such as skyrmionium in chiral magnets. \cite{Fujita_2017} On the other hand, when we apply THz vortex beams to magnets, the direct coupling between the spins and the electromagnetic fields becomes relevant and richer spin dynamics occur. We show from the LLG simulation that the magnetic resonance driven by THz vortex beam in usual ferromagnets leads to a nanostructure with a finite spin scalar chirality, \cite{Fujita_2017a} and it indicates the possibility of vortex-beam driven Hall effect in metallic magnets. We also show that THz vortex beam can simultaneously generate multiple skyrmions in chiral ferromagnets, depending on the value of the angular momentum, \cite{Fujita_2017a} if the beams are sufficiently focusing beyond its diffraction limit with near-field techniques. In the talk, I would like to carefully report the essential aspects of our proposal.